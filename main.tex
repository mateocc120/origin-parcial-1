\documentclass{article}
\usepackage[utf8]{inputenc}
\usepackage[spanish]{babel}
\usepackage{listings}
\usepackage{graphicx}
\graphicspath{ {images/} }
\usepackage{cite}

\begin{document}

\begin{titlepage}
    \begin{center}
        \vspace*{1cm}
            
        \Huge
        \textbf{Calistenia}
            
        \vspace{0.5cm}
        \LARGE
        Parcial 1 - Trabajo
            
        \vspace{1.5cm}
            
        \textbf{Mateo Cardona Correa}
            
        \vfill
            
        \vspace{0.8cm}
            
        \Large
        Despartamento de Ingeniería Electrónica y Telecomunicaciones\\
        Universidad de Antioquia\\
        Medellín\\
        Marzo de 2021
            
    \end{center}
\end{titlepage}

\tableofcontents
\newpage
\section{Introducion al trabajo}\label{intro}
En este trabajo exploraremos la utilizacion de instrucciones para funciones basicas tales como la recolocacion de objetos

\section{Instrucciones} \label{contenido}
En esta parte del documento estaran elaboradas las instrucciones del ejercicio:
INSTRUCCIONES:
1.	Con una sola mano, arrastre la hoja hacia la derecha o hacia la izquierda, según usted convenga, sin dejar caer la hoja de la mesa, hasta dejar a las tarjetas visibles.
2.	Con la misma mano agarre ambas tarjetas alineadas y póngalas en posición vertical encima del centro de la hoja
3.	Sin soltar las tarjetas, ubique el dedo pulgar en una de las esquinas superiores de la tarjeta, mientras que ubica el dedo índice en la otra esquina superior
4.	Ubique dos dedos de la misma mano en uno de los dos bordes de la tarjeta mas cercana a la mano que se está usando.
5.	Sin separar los dedos de las tarjetas y de manera paciente y despacio, vaya separando las tarjetas hasta que se encuentren a unos dos falanges de distancia entre las tarjetas
6.	Retire lentamente los dos dedos puestos en uno de los dos bordes de la tarjeta más cercana a la palma de la mano usada, 
7.	Finalmente con los dedos puestos en la parte superior del triangulo, trate de buscar un equilibrio tal que la estructura se encuentre en sostenida por su propia cuenta y en equilibrio
8.	Si se caen, repita desde el paso 3 en las instrucciones.

\subsection{Estado inicial}
Estado Inicial: Una hoja de papel encima de dos tarjetas del mismo tamaño y grosor

\end{document}
